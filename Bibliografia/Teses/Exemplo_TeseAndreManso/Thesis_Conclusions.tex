%%%%%%%%%%%%%%%%%%%%%%%%%%%%%%%%%%%%%%%%%%%%%%%%%%%%%%%%%%%%%%%%%%%%%%%%
%                                                                      %
%     File: Thesis_Conclusions.tex                                     %
%     Tex Master: Thesis.tex                                           %
%                                                                      %
%     Author: Andre C. Marta                                           %
%     Last modified :  2 Jul 2015                                      %
%                                                                      %
%%%%%%%%%%%%%%%%%%%%%%%%%%%%%%%%%%%%%%%%%%%%%%%%%%%%%%%%%%%%%%%%%%%%%%%%

\chapter{Conclusions}
\label{chapter:conclusions}

The present work was divided into four main stages, 1) the implementation of the data collection system and all its required components, 2) designing, choosing and integrating wearable sensors to allow hospital testing, 3) assessing the system concerning robustness of data collection, device battery life and software performance and 4) a user directed test, with sensor data quality, comfort for patients, ease of use and integration in hospital environment being the main points to be assessed.

The first stage of software implementation was centered into developing all the user interfaces and software components, with versatility and user-friendliness being the key-words.

The second stage was somewhat more challenging, as the choice, design and construction of the wearable sensors to be used in later tests had to cover a large range of requirements. On one side, it should collect relevant medical information for the test conditions, which were to have the system monitor cardiac patients. On the other hand, aspects like battery life, comfort and robustness to wear and tear should also be taken into consideration. The final aspect that proved decisive was the future developments of the system, which motivated the integration of a smart-device into the sensors array. This was intended to allow for other functionalities to be implemented in this smart device.

Following the implementation and development stage, the first tests started. The main focus of this stage was to ensure there were no data loses, bugs, and ensuring the system was easy to use and provided relevant information. Conclusions from all these tests led to some iterations, revealing some software aspects that had to be improved. The final result was a lossless data acquisition system with a very convenient interface that could interact with a great variety of sensors with minimum implementation overhead. These tests also revealed some less positive aspects concerning battery life and data quality. Regarding battery life, it was concluded that it was rather short for what was expected, this led to some modifications in the devices used, with a bigger battery and a power bank being introduced for chest band sensor and smartphone, respectively. Also some software optimizations were implemented in order to reduce the computational burden. Regarding sensors' reliability, issues were detected when looking at the PPG data collected by the smartwatch. The sensor was very prone to noise and motion artifacts, rendering the data unfit for a rigorous medical context. This is one of the main points to address, that can lead to the replacement of the chosen device, or the elaboration of better signal processing algorithms.

Finally, the system was tested in a real life scenario, being used to monitor several patients. The main aspects in analysis were the ease of use by patients and medical teams. Battery life and charging schedule, together with the previously detected lack of reliability from one of the sensors, were the less positive aspects pointed out by medical teams. Although, both this issues could be easily solved by replacing the low performing sensors and designing a device charging schedule, making these aspects minor drawbacks for the system as a whole.

Overall, very positive feedback arose from this last stage. Interfaces and devices were considered easy to use and comfortable by patients and medical teams, providing very useful information and contributing to an increased ability to better diagnose and treat patients. Versatility was one of the most prized aspects, as it allows the system to be used with a vast horizon of medical conditions, and contexts, making timeless, in a sense, as the same system can be used as sensor technology evolves. All medical personnel involved in this testing phase considered this to be a valuable addition to their practice.

An unexpected outcome of this work had to do with patients reaction to the system. Besides considering the system comfortable to wear with minimum disturbance of their daily-life, patients reported to feel more accompanied and better taken care knowing that they are being monitored 24hr a day. This was not expected initially and may contribute to even better outcomes, as feeling more taken care of may improve patients' health status acting similarly to the placebo effect.



% ----------------------------------------------------------------------
\section{Achievements}
\label{section:achievements}

The main objectives of the present work were the development, testing and implementation of a pervasive monitoring system for non-hospitalized patients. The system's features should include support for various sensors, ensuring versatility, and an  easy to use web interface that allows for remote real time data visualization to be integrated as a diagnose and follow up tool for medical teams. Additionally, it was intended that the final system would be tested and integrated into a final user scenario in \ac{hsm}.

After all the work and tests, the outcome of this work is as previously intended: a fully functional pervasive monitoring system that can be used with a great variety of sensors, making it suitable for almost any long term monitoring requirement, with real time data visualization for medical teams witha convenient interface. Alongside with visualization, there is also the possibility to configure alarms when specific conditions occur. The system had very positive feedback from medical teams, that considered it as a very useful diagnosis and follow-up tool, and also patients who considered the system to be comfortable to use and felt they were being given better care as the doctors could observe them continuously, and not sporadically, as happens in traditional hospital health care.


% ----------------------------------------------------------------------
\section{Future Work}
\label{section:future}

As discussed before, the single most important aspect still to address is the battery life of the android smartphone. Despite being the major point that could lead to the misfit of this device to a hospital environment, this may not be easy to solve, as BT connection to the remote sensors, and the signal processing algorithms require a lot of computations making it hard to optimize the software into extending battery life.

Adding to this, there is the specific warable sensors chosen to the cardiology department testing phase. One of the selected wearale sensors, the smartwach, revealed not to be reliable enough to be integrated into this type of monitoring system. This would require this specific sensors to be improved, replaced, or alternatively, that a super-efficient algorthm was used to process the acquired data.

Being the system completely versatile in what sensors it uses, possible improvements on the system are always possible with the implementation of support for newer and better sensors, which could even include environmental sensors. The system was designed to deal with wearable sensor platform constantly collecting data on the patient's physiological parameters. Although other types of sensors could be considered, and integrated into the system, like cameras to allow for motion detection, activity and environmental monitoring, or even devices periodically to monitor patients weight or arterial pressure. Although these types of sensors do not comply with pervasive monitoring, they could still be integrated and provide useful information to medical teams.

A possible addition for this system could be an interface for family and caregivers, allowing them to also be informed about the health state of the patient and even receive notifications when a relevant event takes place e.g. when a patient falls. Which leads to another possible improvement, which would be the detection of events that could be detected through the accelerometer. These events could be simply fall detection or could evolve to full activity recognition. This still presents many challenges elated with computational burden and battery exhaustion, and also with sensors placement, as activity recognition is still a search topic and thus a good and efficient way of accomplishing it may be a challenge to implement in practice.

A final thought on a possible improvement, would be to incorporate sporadically inquiries to the patient as a mean to retrieve the self reported health state or other relevant information.

