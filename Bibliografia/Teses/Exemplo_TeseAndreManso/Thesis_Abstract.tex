%%%%%%%%%%%%%%%%%%%%%%%%%%%%%%%%%%%%%%%%%%%%%%%%%%%%%%%%%%%%%%%%%%%%%%%%
%                                                                      %
%     File: Thesis_Abstract.tex                                        %
%     Tex Master: Thesis.tex                                           %
%                                                                      %
%     Author: Andre C. Marta                                           %
%     Last modified :  2 Jul 2015                                      %
%                                                                      %
%%%%%%%%%%%%%%%%%%%%%%%%%%%%%%%%%%%%%%%%%%%%%%%%%%%%%%%%%%%%%%%%%%%%%%%%

\section*{Abstract}

% Add entry in the table of contents as section
\addcontentsline{toc}{section}{Abstract}

In recent years, sensors and data collection devices have become an ever more ubiquitous presence. This has a major potential concerning medical procedures, opening the path for personalized medicine and allowing for a more targeted and efficient diagnose and therapeutic.

In this work a novel pervasive monitoring system is proposed. It is designed to allow constant and long-term acquisition of any desired variables and visualize the signals remotely in real time via a web interface. The system is implemented to be completely versatile concerning the sensors it can interact with.

System’s functioning is based in two components, a smartphone app and a central server. The smartphone is responsible for connecting with all sensors via Bluetooth and processing, storing and relaying all data to the server. The later is responsible for long-term storing of the data, and for serving a web interface where acquisition parameters can be configured and data can be visualized through an interactive view.

The system was tested in collaboration with cardiology department of Hospital de Santa Marta – CHLC. Tests consisted in utilizing the system with two devices, a BITalino and a Samsung Gear S3 smartwatch, to monitor patient’s heart rate and physical activity, being patient’s comfort and system’s scalability important factors in device choice.

The conducted tests allowed to get feedback from all intended users: patients, physicians and medical staff. They all reported the system as being easy to interact with and very useful in getting medically relevant data, with the easy inclusion of new sensors as a major advantage.



\vfill

\textbf{\Large Keywords:} Personalized Medicine, Pervasive Monitoring, Wearable sensors, Smart devices, Bio-signals

